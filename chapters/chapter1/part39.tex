\subsection{石铸的剑}

张思凡虽然没有像苏雨晴那样跪拜,但也双手合十默念了几句清心咒,却并没有许什么愿,只是默念了几句佛祖保佑,便抬起了头。

“话说,我不是佛教信徒,应该没用吧?”

“佛教不像基督教,不是信徒也可以祈求保佑啦。”苏雨晴笑着说道,“心诚则灵。”

“可是,你难道就不希望父母认可你吗?可是佛祖,保佑你了吗?”

“嗯……唔……这个是……磨难啦……历经磨难之后……一定会有好事发生的。”苏雨晴刚开始还有些支支吾吾的,说到后面却又坚定了下来。

“磨难之后,一定就会有好事吗?好人,真的就一定能去佛祖的净土吗?其实那些都不存在吧……”张思凡叹了口气,像是在喃喃自语,又像是在质问苏雨晴。

苏雨晴咬了咬嘴唇,对于张思凡的这番话并没有生气,或许,就连她自己,都不怎么相信那些东西的存在吧。

“心灵的寄托吧……”苏雨晴也幽幽地叹了口气,算是回答了张思凡的问题。

“走吧?”

“嗯,走吧。”

苏雨晴最后看了一眼这尊脸部都已经模糊不清了的佛像,隐约觉得它似乎并不是自己曾经见过的任何一尊佛,难道说,这其实并不是佛像吗?

“咦,等等。”苏雨晴停住了脚步,她指了指佛像后面的一扇小门,那里本来应该算是暗门,只是因为年久失修,已经不是很牢固了,所以能让苏雨晴很明显地发现了它和墙体的区别。

“怎么了?”

“那里有一扇暗门诶。”

“暗门?”张思凡有些疑惑地顺着苏雨晴的视线望去,看到了那扇被风吹得有些晃动的小门,“诶?这里竟然有一扇门,我还从来不知道呢。”

“进去看看吧?”

“好呀。”

苏雨晴和张思凡推门走了进去,门后其实没有什么东西,只是一个很狭小的房间,看起来有些像杂物间,只是在这个“杂物间”的正中央,供奉着一把石头做的剑,看它的雕刻手法,和外面的雕像很相近,或许是同一个人或者同一批人雕刻的吧。

而且因为放在这间比较封闭的屋子里,石头剑的磨损并不算特别严重,甚至就连剑身上所雕刻的字都能让人看得一清二楚。

这是用小篆书写的“轮回”二字,因为这两个字比较简单,而且接近现在用的繁体字,所以张思凡只是一会儿就辨认出来了。

张思凡是因为喜欢看和历史有关的书籍,连带着也记下了不少的古人用的字,而苏雨晴则是因为外婆家在台湾,虽然繁体字会写得不多,但是基本都能看得懂,也是很快就辨认出了这两字。

“轮回……?”苏雨晴有些疑惑地伸手摸了摸剑身上刻着字体处的凹痕,凹凸不平的感觉让人有一种时光流逝的错觉,“我记得,佛教里应该只有四大天王是用剑的,但是也没有一个是用一柄叫做‘轮回’的剑呢……”

“或许是某个比较隐秘的,知名度不是很高的佛教人物吧。”

“可能是吧,‘轮回’这两个字,倒是有些挺符合佛教的取名风格的……”

这把剑虽然是用石头雕刻而成的,但依然能让人感觉到一股杀伐之气,难以想象,那把真正的“轮回剑”到底有多么的杀意冲天,而使用那把剑的人,又杀了多少生灵呢?

在这个小房间最里面,还有一扇门,打开后就是小寺庙外边的走廊了,这条走廊紧挨着山体边缘,虽然有木制的栏杆拦着,但是从上往下看,还是让人有一种神晕目眩的感觉。

“好、好好好高……”有些恐高的苏雨晴站在如此贴近山体边缘的地方,仿佛只要脚一滑就能掉下去,顿时感觉双腿都有些发软了。

在百流山的其他地方都是一个陡坡,虽然也很高,但不然这里给苏雨晴的冲击大,因为这里是一处悬崖,坡度是九十度,可以想象一下站在百层高楼上向下看,大概就是这样的感觉吧……

苏雨晴下意识地紧抓着张思凡的手臂,整个人都有些微微颤抖。

张思凡看着如同小猫一样几乎都快钻到自己怀里来的苏雨晴,有些忍不住将她直接抱进了怀里。

“哇啊啊——”苏雨晴被吓了一跳,顿时惊慌失措地大喊了起来。

“安啦安啦,又不是把你推下去。”张思凡狡黠地笑着,在苏雨晴的耳朵旁吹着气,“小晴,你好可爱诶,就像小猫咪一样~”

“诶、诶诶?”被张思凡往耳朵里吹气的苏雨晴顿时觉得浑身酥软,小脸也有些微微发红了,虽说张思凡也是同类人,但最起码他现在是男装,还是个挺帅的男孩子的模样……

苏雨晴并拢了双腿,这种燥热的感觉让她感到十分的害羞。

“好啦好啦,既然你这么害怕,那我们就下山去吧。”

“嗯……”苏雨晴有些尴尬地点了点头。

下山从消耗的体力上来说,可能会轻松一点,但是一点都不比上山容易,因为下山还要控制走下阶梯时的力道,要是用力过猛,说不定会直接从山坡上滚下去呢……

二人慢慢地走着,倒是没有再休息,因为下山确实不算累,如果能控制好力道,还可以跑着下去,奔跑时冷风拂过脸庞的感觉,会让人觉得更加舒服呢。

“思思姐,你家住在哪里?”

“在这里的吗?”

“是呀。”

“住在郊区哦~”

“不住在学校里吗?”

“住宿舍?那怎么行……很多东西不是都要被看到了嘛。”

“姆,也对哦……”

虽然下山比较不耗费体力,但是等爬到山下的时候,双腿都已经酸痛发麻,甚至感觉都有点僵硬了。

“好累……”苏雨晴双手撑在膝盖上,大口地喘着气,眼前不远处就是一个公交车站,车站里没有一个人,想来星期一也不会有什么人到这种地方来吧。

“比上山轻松多啦。”张思凡从口袋里掏出一包湿巾纸,抽出一张递给苏雨晴,道,“擦擦汗吧,黏在身上很难受的呢。”

“唔,谢谢。”

“客气什么啦。”张思凡挥了挥手,对于苏雨晴这样的见外有些不满,“我们在网络上是最好的朋友,在现实里也一定会是的哦。”

“啊……嗯……”

从百流山到张思凡的家其实不算特别远,有一路公交车直达的。

苏雨晴将车窗拉开,眯着眼睛悠闲地吹着风,虽然座椅是塑料硬座,但依然让她觉得无比的舒服,特别是公交车轻微的颠簸,更是让苏雨晴有一种回到了小时候睡在摇篮里的感觉。

“思思姐也是住在农民房里吗?”苏雨晴眯着眼睛看着四周不断倒退的风景,感觉公交车越往前开,路就越荒凉,四周都是大块大块的农田,以及点缀在其间的民居。

“不是哦~”张思凡神秘地笑了笑,“到了你就知道啦。”

“嗯?难道是普通的居民房?”苏雨晴有些好奇了,可是这后面一片应该都是农村才对,怎么说也不会有城市里的那种小区才对吧?

公交车很快就到达了终点站,这里的终点站和城里面的那个是截然不同的两种风格,城里面的终点站外面围了一圈的小贩,还有各种各样的商铺,而这个终点站,就真的只是一个公交车站而已,一块空地,用水泥墙圈起来,中间有一幢三层的小房子作为这里的办公楼,外面只有一家看起来脏兮兮的快餐店……

“走啦,小晴,我们骑电瓶车过去。”张思凡说着,拉着苏雨晴走出了这个有些“荒凉”的终点站,在外面停自行车和电瓶车的地方把自己的电瓶车开了出来。

一辆飞利浦的电瓶车,在这个年代,飞利浦电瓶车就是口碑的保证,而且经久耐用,价格也挺实惠,受到许多人的欢迎。

这种电瓶车本就是有些像摩托车的,准确的说应该是电动摩托车,载下两个人那是一点问题都没有,更何况苏雨晴和张思凡都属于比较瘦小的,就算三个人也能坐得下吧?

“思思姐的家很远吗?”

“有点距离吧。”张思凡回转了转把手,道,“小晴,抱住我,不要掉下去哦。”

“嗯……”苏雨晴双手环抱住张思凡的腰,将小脸贴在他的背脊上,就像是一只黏人而乖巧的小猫一样。

“哈哈——小晴,别用脸蹭我的背呀,好痒——”张思凡有些忍不住笑了起来,就连车把手都有点握不稳了。

“这样……可以吗?”

“可以可以,大概十五分钟左右就能到了,我们开快点~”

电动车在山村小路里疾驰着,两边的农田有不少农民在忙活着,有些早播下去的种子,此刻已经发了芽,将农田染上了一片翠绿的颜色。

前面的路越来越破,刚开始好歹还是水泥铺过的路,后面就完全是石子小路了,颠簸得厉害,苏雨晴感觉自己的屁股都快被震麻了。

很快,前方就出现了一片草地,在草地后面是一片茂密的树林,或者说是一座小森林更为合适吧。

这里已经没有农田了,只在嫩绿的草地上建造着一座造型怪异的房子,与其说是房子,不如说是集装箱更为合适呢。

……
